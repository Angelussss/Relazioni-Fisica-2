\documentclass[letterpaper, 9pt]{extarticle}
% \usepackage{fontspec}

% ==================================================

% document parameters
% \usepackage[spanish, mexico, es-lcroman]{babel}
\usepackage[english]{babel}
\usepackage[margin=2cm]{geometry}

% ==================================================

% Packages for math
\usepackage{mathrsfs}
\usepackage{amsfonts}
\usepackage{amsmath}
\usepackage{amsthm}
\usepackage{amssymb}
\usepackage{physics}
\usepackage{dsfont}
\usepackage{esint}

% ==================================================

% Packages for writing
\usepackage{enumerate}
\usepackage[shortlabels]{enumitem}
\usepackage{framed}
\usepackage{csquotes}
\usepackage{gensymb}

% ==================================================

% Miscellaneous packages
\usepackage{float}
\usepackage{tabularx}
\usepackage{xcolor}
\usepackage{multicol}
\usepackage{subcaption}
\usepackage{caption}
\usepackage{graphicx}
\usepackage{array}
\usepackage{circuitikz}
\captionsetup{format = hang, margin = 10pt, font = small, labelfont = bf}
\usepackage{pgfplots} 

% Citation
\usepackage[round, authoryear]{natbib}

% Hyperlinks setup
\usepackage{hyperref}
\definecolor{links}{rgb}{0.36,0.54,0.66}
\hypersetup{
   colorlinks = true,
    linkcolor = black,
     urlcolor = blue,
    citecolor = blue,
    filecolor = blue,
    pdfauthor = {Author},
     pdftitle = {Title},
   pdfsubject = {subject},
  pdfkeywords = {one, two},
  pdfproducer = {LaTeX},
   pdfcreator = {pdfLaTeX},
   }

\usepackage{titlesec}
\usepackage[many]{tcolorbox}

% Adjust spacing after the chapter title
\titlespacing*{\chapter}{0cm}{-2.0cm}{0.50cm}
\titlespacing*{\section}{0cm}{0.50cm}{0.25cm}

% Indent 
\setlength{\parindent}{0pt}
\setlength{\parskip}{1ex}

% --- Theorems, lemma, corollary, postulate, definition ---
% \numberwithin{equation}{section}

\newtcbtheorem[]{problem}{Problem}%
    {enhanced,
    colback = black!5, %white,
    colbacktitle = black!5,
    coltitle = black,
    boxrule = 0pt,
    frame hidden,
    borderline west = {0.5mm}{0.0mm}{black},
    fonttitle = \bfseries\sffamily,
    breakable,
    before skip = 3ex,
    after skip = 3ex
}{problem}

\tcbuselibrary{skins, breakable}

% --- You can define your own color box. Just copy the previous \newtcbtheorm definition and use the colors of yout liking and the title you want to use.
% --- Basic commands ---
%   Euler's constant
\newcommand{\eu}{\mathrm{e}}

%   Imaginary unit
\newcommand{\im}{\mathrm{i}}

%   Sexagesimal degree symbol
\newcommand{\grado}{\,^{\circ}}

% --- Comandos para álgebra lineal ---
% Matrix transpose
\newcommand{\transpose}[1]{{#1}^{\mathsf{T}}}

%%% Comandos para cálculo
%   Definite integral from -\infty to +\infty
\newcommand{\Int}{\int\limits_{-\infty}^{\infty}}

%   Indefinite integral
\newcommand{\rint}[2]{\int{#1}\dd{#2}}

%  Definite integral
\newcommand{\Rint}[4]{\int\limits_{#1}^{#2}{#3}\dd{#4}}

%   Dot product symbol (use the command \bigcdot)
\makeatletter
\newcommand*\bigcdot{\mathpalette\bigcdot@{.5}}
\newcommand*\bigcdot@[2]{\mathbin{\vcenter{\hbox{\scalebox{#2}{$\m@th#1\bullet$}}}}}
\makeatother

%   Hamiltonian
\newcommand{\Ham}{\hat{\mathcal{H}}}

%   Trace
\renewcommand{\Tr}{\mathrm{Tr}}

% Christoffel symbol of the second kind
\newcommand{\christoffelsecond}[4]{\dfrac{1}{2}g^{#3 #4}(\partial_{#1} g_{#2 #4} + \partial_{#2} g_{#1 #4} - \partial_{#4} g_{#1 #2})}

% Riemann curvature tensor
\newcommand{\riemanncurvature}[5]{\partial_{#3} \Gamma_{#4 #2}^{#1} - \partial_{#4} \Gamma_{#3 #2}^{#1} + \Gamma_{#3 #5}^{#1} \Gamma_{#4 #2}^{#5} - \Gamma_{#4 #5}^{#1} \Gamma_{#3 #2}^{#5}}

% Covariant Riemann curvature tensor
\newcommand{\covariantriemanncurvature}[5]{g_{#1 #5} R^{#5}{}_{#2 #3 #4}}

% Ricci tensor
\newcommand{\riccitensor}[5]{g_{#1 #5} R^{#5}{}_{#2 #3 #4}}
\begin{document}
\setlist{itemsep=.5em}
\begin{Huge}
\textsf{\textbf{Fisica 2 (Teoria dei Circuiti)}}\\
Esperienza 3 (tutto il contenuto e' da modificare)
\end{Huge}

\vspace{1ex}

\textsf{\textbf{Studenti:}} \text{Angelo Perotti},  \text{Mattia Zagatti}, \text{Mattia Dolci}

\vspace{2ex}

\section{Introduzione}
In questa esperienza viene analizzato il comportamento del circuito passa banda RLC, osservando il suo
andamento prima nel dominio della frequenza attraverso lo sviluppo dei diagrammi di Bode e poi in quello
del tempo.

\section{Materiale utilizzato}
\begin{itemize}
    \item \textbf{Componenti elettronici}: resistori (\(1 \, \text{k}\Omega\), \(10 \, \text{k}\Omega\)), capacitori (\(1\text{nF}\) (\(10 \text{nF}\), \(100\text{nF}\)), decade di induttanze, breadboard.
    \item \textbf{Strumenti di misura}: generatore di forme d'onda, oscilloscopio
    \item \textbf{Cavi}: cavi bnc, cavi banana-banana, cavi jumper.
\end{itemize}

\section{circuito utilizzato}
\begin{figure}[!ht]
\centering %faccio un pane e nutella e torno %tranquillo amio <- cimelio storico
\resizebox{0.45\textwidth}{!}{%
\begin{circuitikz}
\tikzstyle{every node}=[font=\large]
\draw (8.75,14.75) to[L ] (11.25,14.75);
\draw (8.75,14.75) to[C] (7.5,14.75);
\draw (11.25,14.75) to[R] (11.25,12.25);
\draw (11.25,14.75) to[short, -o] (12.5,14.75) ;
\draw (11.25,12.25) to[short, -o] (12.5,12.25) ;
\draw (11.25,12.25) to (11.25,12) node[ground]{};
\draw (7.5,12.25) to[short] (11.25,12.25);
\draw (7.5,12.25) to[sinusoidal voltage source, sources/symbol/rotate=auto] (7.5,14.75);
\node [font=\medium] at (7,13.75) {+};
\node [font=\medium] at (7,13.25) {-};
\node [font=\large] at (6.25,13.5) {$V_{in}(t)$};
\node [font=\large] at (8.15,15.4) {C};
\node [font=\large] at (10,15.25) {L};
\node [font=\large] at (10.5,13.5) {R};
\node [font=\large] at (13,13.5) {$V_{out}(t)$};
\node [font=\large] at (13,14.75) {+};
\node [font=\large] at (13,12.25) {-};
\end{circuitikz}
}%

\label{fig:my_label}
\end{figure}

\section{Esperimento 1}

In questo esercizio lo scopo è quello di analizzare il circuito nel dominio della frequenza, realizzando i relativi
diagramma di Bode del modulo e della fase al variare della resistenza presente nel circuito. Esso è alimentato
da un segnale sinusoidale di ampiezza picco-picco pari a 5 V e offset 0 V, ottenuto mediante un generatore di
segnali collegato opportunamente alla breadboard mediante gli appositi doppietti. Gli altri componenti sono
un resistore di valore 10 k$\ohm$ nel primo caso e 1 k$\ohm$ nel secondo, un condensatore di 10 nF e un induttore di
500 mH. Questo valore di induttanza è stato introdotto all’interno del circuito non attraverso il classico bipolo
ma con una decade di induttanza, regolando la manopola relativa all’ordine di grandezza appropriato.

$$\huge R= 10k\ohm$$

\begin{figure}[ht]
\centering
\begin{minipage}{0.5\textwidth}
\centering
\vspace{0pt}
    \centering
    \begin{tabular}{c|c|c|c}
        f & $\Delta$ & A_{ing} & A_{usc}\\
        \hline
        1Hz & 84\degree & 5V & 4mV \\
        \hline
        1,15KHz & 45,3\degree & 5V & 3,34V \\
        \hline
        2Hz & 92\degree & 5V & 6,25mV\\
        \hline
        4,5V & 47\degree & 5V & 3,38V \\
        \hline
        5Hz & 91,4\degree & 5V & 15,73mV \\
        \hline
        10Hz & 89,8\degree & 5V & 30,4mV \\
        \hline
        20Hz & 89\degree & 5V & 60,55mV \\
        \hline
        50Hz & 87\degree & 5V & 152mv \\
        \hline
        100Hz & 86\degree & 5V & 304,2mV \\
        \hline
        200Hz & 81\degree & 5V & 602,5mV \\
        \hline
        500Hz & 70\degree & 5V & 1,5V \\
        \hline
        1KHz & 51\degree & 5V & 2,93V \\
        \hline
        2KHz & 9,8\degree & 5V & 4,69V \\
        \hline
        2,25Khz & 0\degree & 5V & 4,77V \\
        \hline
        5Khz & -53,4\degree & 5V & 3,07V \\
        \hline
        10Khz & -77\degree & 5V & 1,5V \\
        \hline
        50KHz & - & 5V & - \\
    \end{tabular}
    \caption{}
    \label{tab:my_label}
\label{fig:my_label}
\end{minipage}%
\begin{minipage}{0.5\textwidth}
\centering
\vspace{0pt}

Utilizzando un valore di R pari a 10 k$\ohm$, le misure sono state effettuate variando il valore di frequenza del
segnale in ingresso da un minimo di 1 Hz a un massimo di 10 kHz. Oltre a queste 13 misure, vengono
considerate altre tre frequenze: i due valori corrispondente ad un guadagno di -3 dB rispetto al suo massimo
e la frequenza di risonanza. Le prime sono state ottenute calcolando il guadagno a -3 dB, ottenuto
considerando il valore massimo di ampiezza del segnale di uscita misurato ai capi del resistore (4.77 V) diviso
la radice di 2:

$$A_{\epsilon dB} = \dfrac{A_{max}}{\sqrt{2}}=\dfrac{4,77V}{\sqrt{2}}=3,37V$$

Successivamente, si regola la frequenza dal generatore di segnali fino a quando non si ottiene un valore di
tensione paragonabile, in questo caso f1 = 1.15 kHz e f2 = 4.5 kHz.
Per quanto riguarda la frequenza di risonanza, essa è stata ricavata considerando prima il valore della
pulsazione in questione e poi dividendo per 2$\pi$:

$$\omega_0=\dfrac{1}{\sqrt{LC}}=14142\dfrac{rad}{s}$$
$$f_0=\dfrac{\omega_o}{2\pi}=2.25kHz$$

Per misurare l’ampiezza e lo sfasamento sull’oscilloscopio sono stati utilizzati gli appositi cursori, variando
opportunamente la scala dei tempi e dell’ampiezza.

 \end{minipage}
\end{figure}

*FOTO DIAGRAMMI DI BODE GUADAGNO E FASE*
I diagrammi di Bode sono stati realizzati utilizzando i dati raccolti in laboratorio (vedi tabella), facendo
attenzione allo sfasamento tra i due segnali ed effettuando la seguente conversione:
\begin{center}
    

$\Delta\O[rad]=\dfrac{\Delta t}{T}\cdot 2\pi$
\end{center}
Come è possibile notare, i risultati ottenuti sono coerenti con il comportamento passa-banda del circuito,
secondo il quale si ha un guadagno massimo( G = $\dfrac{A_{out}}{A_{in}}$) in corrispondenza della frequenza di risonanza,
mentre il segnale di uscita viene particolarmente attenuato a frequenze maggiori rispetto a 10 $f_0$ e inferiori rispetto a 0.1 $f_0$ .

r=1k
In questo caso vengono ripetute le misure precedenti modificando il valore di R, andando perciò a variare il
valore di tensione massima e di conseguenza le frequenze corrispondenti ad un valore di ampiezza inferiore
di 3 dB rispetto a quello massimo, mentre la frequenza di risonanza rimane invariata in quanto i valori di
capacità e induttanza sono invariati. Le misure a 1 e 2 Hz non sono state effettuate in quando il segnale di
uscita in quel punto era molto attenuato e non era quindi osservabile all’oscilloscopio.


$$A_{3dB}=\dfrac{3.37V}{\sqrt{2}}=2.38V$$
foto diagrammi bode

Rispetto al caso precedente si nota come i valori di tensione misurati non rispecchiano perfettamente quelli
attesi, in particolar modo il valore massimo del segnale in uscita (3.37 V) è decisamente inferiore rispetto a
quello misurato con R = 10 k$\omega$ (4.77 V), e di conseguenza anche i valori relativi allo sfasamento. Ciò è causato
dal fatto che, avendo in questo caso un valore di resistenza più basso, esso è paragonabile al valore della
resistenza parassita presente in serie all’induttore a causa del suo comportamento reale


\begin{figure}[!ht]
\centering %faccio un pane e nutella e torno %tranquillo amio <- cimelio storico
\resizebox{0.45\textwidth}{!}{%
\begin{circuitikz}
\tikzstyle{every node}=[font=\large]
\draw (8.75,14.75) to[L ] (10,14.75);
\draw (8.75,14.75) to[C] (7.5,14.75);
\draw (11.25,14.75) to[R] (11.25,12.25);
\draw (11.25,14.75) to[short, -o] (12.5,14.75) ;
\draw (11.25,12.25) to[short, -o] (12.5,12.25) ;
\draw (11.25,12.25) to (11.25,12) node[ground]{};
\draw (7.5,12.25) to[short] (11.25,12.25);
\draw (7.5,12.25) to[sinusoidal voltage source, sources/symbol/rotate=auto] (7.5,14.75);
\node [font=\medium] at (7,13.75) {+};
\node [font=\medium] at (7,13.25) {-};
\node [font=\large] at (6.25,13.5) {$V_{in}(t)$};
\node [font=\large] at (8.15,15.4) {C};
\node [font=\large] at (9.4,15.25) {L};
\node [font=\large] at (10.5,13.5) {R};
\node [font=\large] at (13,13.5) {$V_{out}(t)$};
\node [font=\large] at (13,14.75) {+};
\node [font=\large] at (13,12.25) {-};
\draw (10,14.75) to[R] (11.25,14.75);
\node [font=\large] at (10.5,15.25) {Ri};
\end{circuitikz}
}%

\label{fig:my_label}
\end{figure}

\label{fig:my_label}
\end{figure}

\centering
    \begin{tabular}{c|c|c|c}
        f & $\Delta$ & A_{ing} & A_{usc}\\
        \hline
        1Hz & - & 5V & - \\
        \hline
        \hline
        2Hz & - & 5V & -\\
        \hline
        5Hz & - & 5V & - \\
        \hline
        10Hz & 88\degree & 5V & 3,3mV \\
        \hline
        20Hz & 89\degree & 5V & 6,3mV \\
        \hline
        50Hz & 90\degree & 5V & 15,3mv \\
        \hline
        100Hz & 89\degree & 5V & 36,6mV \\
        \hline
        200Hz & 88\degree & 5V & 61mV \\
        \hline
        500Hz & 86,7\degree & 5V & 161mV \\
        \hline
        1KHz & 83\degree & 5V & 373mV \\
        \hline
        2KHz & 48\degree & 5V & 4,69V \\
        \hline
        2,25Khz & 0\degree & 5V & 4,77V \\
        \hline
        5Khz & -53,4\degree & 5V & 3,07V \\
        \hline
        10Khz & -77\degree & 5V & 1,5V \\
        \hline
        50KHz & - & 5V & - \\
    \end{tabular}
    \caption{}
    \label{tab:my_label}



\end{document}